\section{Jak Texat--Vzorová otázka} %Název podotázky, přesně tak, jak se to má vytexat

\begin{shrnuti}
Pište v \LaTeX u, snažte se nedělat moc chyb v syntaxi, prosím. Soubor s otázkou pojmenovávejte \verb"#.tex", kde zadejte číslo podle tabulky na drivu.
(Sem patří stručné shrnutí otázky, na jeden dva odstavce maximálně, fakt věci, které když nebudete znát, tak poletíte oknem.)
\end{shrnuti}

\subsection*{Jak na \LaTeX}
To už snad většina lidí zná, pokud ne, je na internetu mnoho návodů. Zde jen nějaké poznámky:

Odkazujte co se dá -- je to pak klikací!!!
\begin{equation}
\label{0_rovnice}
\int_0^d d \d d = \frac{d^2}{2}
\end{equation}
V rovnici \eqref{0_rovnice} je \verb"\label{0_rovnice}", 
aby odkaz pak byl pomocí \verb"\eqref{0_rovnice}"
Diferenciál pište pomocí \verb"\d".\footnote{je to makro definované jen v této šabloně, 
jinde vám to fungovat nebude\dots}
Dá se odkazovat na obrázky, tabulky, kapitoly a kdovíco ještě. Stačí, když dovnitř dáte 
\verb"\label{identfikátor}" s jedinečným identifikátorem a pak vás 
\verb"\ref{identifikátor}" odkáže Používejte identifikátory ve tvaru \verb"#_něco", kam 
dáte číslo otázky, aby se předešlo kolizím napříč otázkami.
Speciálně, pokud chcete 
se odkázat na jinou otázku, použijte \verb"\ref{ot#}", kde zadáte číslo souboru otázky. 
(Např. \verb"\ref{ot42}" vás odkáže na otázku \ref{ot42}).
\pojem{Pojem} se zvýrazní pomocí \verb"\pojem{To co chcete zvýraznit}"\footnote{Tady 
vám snad došlo, že je to další makro. :)} 
\subsubsection*{Odkazy na zdroje}
Odkazy na literaturu pomocí bib\TeX u, ještě se bude řešit.
Odkazy na webově stránky pak pomocí \verb"\url{www.whatever.com}" -- taky je to pak klikací.

\begin{poznamky} %upravený itemize, používat jako itemize
\item Odkud se dá čerpat z literatury, na co se doptávají klasicky u státnic, na co se dá navázat, s čím to souvisí apod. Takový tipy a triky.
\item Odkaz na rovnici \eqref{0_rovnice} fakt je klikací.
\item Žádný učený z nebe nespadl, nebojte se zeptat, pokud vám něco nejde, leckdo z ročníku \TeX at umí a rád vám poradí. (třeba já)
\item Chcete-li si ztexat jen svou otázku, abyste zjistili, jestli nemáte v ní chyby,  stáhněte soubor \verb"ukazka.tex" a v příkazu \verb"\otazka{}" přepište \uv{vzor} na soubor své otázky bez přípony. (v ideálním případě by to mělo být jen číslo) 
\item Tento vzor bude asi průběžně doplňován.
\item Výsledek bude vytvářen spíš na Githubu \url{https://github.com/matoumik/bc-statnice}
\item Bývá dobrým zvykem, zalamovat žádky po cca 80 znacích (cca šířka obrazovky)
\end{poznamky}

\autori{Mikuláš Matoušek}{většinu definic maker jsem ukradl ze stack overflow}
% Makro sází autory v prvním parametru (zase, jak se to napíše, 
%tak se to vytexá, víc autorů 
%oddělte čárkami) v druhém parametru je
%pak místo pro poznámku autora (odkaď to ukradl, cokoliv jej nepadne)