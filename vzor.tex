\section{Vzorová otázka} %Název podotázky, přesně tak, jak se to má vytexat

\begin{shrnuti}
Sazba tohoto se asi ještě dooslí -- Stručné shrnutí otázky
\end{shrnuti}

\subsection*{Tady pak skutečně zpracujte otázku}
\lipsum[1]
\subsubsection*{Případně podnadpis}
\lipsum[1]
\subsection*{Whatever}
Amen

\begin{poznamky} %upravený itemize, používat jako itemize
\item Odkud se dá čerpat z literatury, na co se doptávají klasicky u státnic, na co se dá navázat, s čím to souvisí apod. Takový tipy a triky. viz \ref{ot99}
\item Tento vzor bude sloužit jako návod k použití, až to dodělám.
\end{poznamky}

\autori{Mikuláš Matoušek}{většinu definic maker jsem ukradl ze stack overflow}
% Makro,
%sází autora v prvním parametru (zase, jak se to napíše, tak se to vytexá, víc autorů % %oddělte čárkami) v druhém parametru je 
%pak místo pro poznámku autora