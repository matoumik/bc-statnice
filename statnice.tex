%%% Hlavní soubor. Zde se definují základní parametry a odkazuje se na ostatní části. %%%

%% Verze pro jednostranný tisk:
% Okraje: levý 40mm, pravý 25mm, horní a dolní 25mm
% (ale pozor, LaTeX si sám přidává 1in)
\documentclass[12pt,a4paper, hidelinks]{report} 
\setlength\textwidth{145mm}
\setlength\textheight{247mm}
\setlength\oddsidemargin{15mm}
\setlength\evensidemargin{15mm}
\setlength\topmargin{0mm}
\setlength\headsep{0mm}
\setlength\headheight{0mm}
% \openright zařídí, aby následující text začínal na pravé straně knihy
\let\openright=\clearpage

%% Pokud tiskneme oboustranně:
% \documentclass[12pt,a4paper,twoside,openright]{report}
% \setlength\textwidth{145mm}
% \setlength\textheight{247mm}
% \setlength\oddsidemargin{14.2mm}
% \setlength\evensidemargin{0mm}
% \setlength\topmargin{0mm}
% \setlength\headsep{0mm}
% \setlength\headheight{0mm}
% \let\openright=\cleardoublepage

%% Vytváříme PDF/A-2u - haha, hele spíš ne než jo
\usepackage[a-2u]{pdfx}

%% Přepneme na českou sazbu a fonty Latin Modern
\usepackage[czech]{babel}
\usepackage{lmodern}
\usepackage[T1]{fontenc}
\usepackage{textcomp}

%% Vlastní knihovny
\usepackage{hyperref}
\usepackage{graphicx}
\usepackage{siunitx}

%% Obrázky
\graphicspath{
	{png/}
	{data/}
}
\usepackage{wrapfig}
\usepackage{float}
\usepackage[caption = false]{subfig}

%% Použité kódování znaků: obvykle latin2, cp1250 nebo utf8:
\usepackage[utf8]{inputenc}

%%% Další užitečné balíčky (jsou součástí běžných distribucí LaTeXu)
\usepackage{amsmath}        % rozšíření pro sazbu matematiky
\usepackage{amsfonts}       % matematické fonty
\usepackage{amsthm}         % sazba vět, definic apod.
\usepackage{bbding}         % balíček s nejrůznějšími symboly
			    % (čtverečky, hvězdičky, tužtičky, nůžtičky, ...)
\usepackage{bm}             % tučné symboly (příkaz \bm)
\usepackage{graphicx}       % vkládání obrázků
\usepackage{fancyvrb}       % vylepšené prostředí pro strojové písmo
\usepackage{indentfirst}    % zavede odsazení 1. odstavce kapitoly
\usepackage{natbib}         % zajištuje možnost odkazovat na literaturu
			    % stylem AUTOR (ROK), resp. AUTOR [ČÍSLO]
\usepackage[nottoc]{tocbibind} % zajistí přidání seznamu literatury,
                            % obrázků a tabulek do obsahu
\usepackage{icomma}         % inteligetní čárka v matematickém módu
\usepackage{dcolumn}        % lepší zarovnání sloupců v tabulkách
\usepackage{booktabs}       % lepší vodorovné linky v tabulkách
\usepackage{paralist}       % lepší enumerate a itemize
\usepackage[usenames]{xcolor}  % barevná sazba

\hypersetup{unicode}
%\hypersetup{breaklinks=true}

\usepackage{forloop}
\newcommand{\otazka}[1]{%
\IfFileExists{#1.tex}{\input{#1.tex}}{\section{\rm\it Otázka ještě čeká na zpracování}}
\label{ot#1}
}
\newcommand{\autori}[2]{%
\hrule\smallskip{\small #1}\\\relax
{
\footnotesize\it(#2)}
}
\newenvironment{poznamky}{{\centering\subsubsection*{Poznámky}}\begin{itemize}}{\end{itemize}}
\newenvironment{shrnuti}{{\centering\subsubsection*{Základy}}}{\bigskip\hrule}

\AtBeginDocument{\renewcommand{\d}{\,\mathrm{d}}}

\newcommand{\pojem}[1]{{\bf #1}}

\usepackage{lipsum}

%% Titulní strana
\begin{document}
\newcounter{ot}

%%% Strana s automaticky generovaným obsahem

\chapter*{Úvodem}
Šablona je funkční, jen ji chybí skutečně zpracované otázky. Jen do toho -- podívejte se do vzorového souboru, jak formátovat soubor s otázkou.

Sazba se dooslí, ale mělo by to být jen úpravou maker

Až toto bude vypadat nějak rozumně, někdo napište skutečný úvod pro budoucí generace.

\newpage

\tableofcontents

%%% Jednotlivé otázky
\newpage
\otazka{vzor}

\chapter{Mechanika hmotného bodu a soustav hmotných bodů}
\forloop{ot}{1}{\value{ot} < 10}{
\otazka{\arabic{ot}}
}
\chapter{Kinematika a dynamika tuhého tělesa}
\forloop{ot}{\value{ot}}{\value{ot} < 14}{
\otazka{\arabic{ot}}
}
\chapter{Mechanika kontinua}
\forloop{ot}{\value{ot}}{\value{ot} < 19}{
\otazka{\arabic{ot}}
}
\chapter{Struktura látek}
\forloop{ot}{\value{ot}}{\value{ot} < 23}{
\otazka{\arabic{ot}}
}
\chapter{Základy termodynamiky a statistické fyziky}
\forloop{ot}{\value{ot}}{\value{ot} < 32}{
\otazka{\arabic{ot}}
}
\chapter{Základy kinetické teorie}
\forloop{ot}{\value{ot}}{\value{ot} < 35}{
\otazka{\arabic{ot}}
}
\chapter{Základní elektromagnetické veličiny a jejich měření}
\forloop{ot}{\value{ot}}{\value{ot} < 39}{
\otazka{\arabic{ot}}
}
\chapter{Maxwellovy rovnice a jejich základní důsledky}
\forloop{ot}{\value{ot}}{\value{ot} < 42}{
\otazka{\arabic{ot}}
}
\chapter{Základní principy speciální teorie relativity}
\forloop{ot}{\value{ot}}{\value{ot} < 47}{
\otazka{\arabic{ot}}
}
\chapter{Elektrické obvody stacionární, kvazistacionární a střídavé}
\forloop{ot}{\value{ot}}{\value{ot} < 51}{
\otazka{\arabic{ot}}
}
\chapter{Elektromagnetické vlny}
\forloop{ot}{\value{ot}}{\value{ot} < 58}{
\otazka{\arabic{ot}}
}
\chapter{Optika}
\forloop{ot}{\value{ot}}{\value{ot} < 72}{
\otazka{\arabic{ot}}
}
\chapter{Variační formulace fyzikálních zákonů}
\forloop{ot}{\value{ot}}{\value{ot} < 74}{
\otazka{\arabic{ot}}
}
\chapter{Stavba atomů, molekul a kondenzovaných látek}
\forloop{ot}{\value{ot}}{\value{ot} < 79}{
\otazka{\arabic{ot}}
}
\chapter{Experimentální základy kvantové hypotézy}
\forloop{ot}{\value{ot}}{\value{ot} < 82}{
\otazka{\arabic{ot}}
}
\chapter{Formalizmus kvantové teorie}
\forloop{ot}{\value{ot}}{\value{ot} < 90}{
\otazka{\arabic{ot}}
}
\chapter{Aplikace kvantové mechaniky}
\forloop{ot}{\value{ot}}{\value{ot} < 95}{
\otazka{\arabic{ot}}
}
\chapter{Jaderné záření}
\forloop{ot}{\value{ot}}{\value{ot} < 98}{
\otazka{\arabic{ot}}
}
\chapter{Atomové jádro}
\forloop{ot}{\value{ot}}{\value{ot} < 102}{
\otazka{\arabic{ot}}
}
\chapter{Subjaderná fyzika}
\forloop{ot}{\value{ot}}{\value{ot} < 104}{
\otazka{\arabic{ot}}
}
\openright
\end{document}
